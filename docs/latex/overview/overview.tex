\section{Overview}

In this section, an overview of the circuit simulator software will be presented. In SPICE, it supported many semiconductor device compact models and elements, including but not limited to BJTs, junction diodes, resistors, capacitors, inductors, independent voltage and current sources.\cite{SPICE} One requirement in the design was that the software is able to deal with basic components, reactive components, such as capacitors and inductors, and AC sources that provide time-varying voltages in the circuit. Some general features in SPICE must be obtained in certain aspects, such as creating functions that are able to give exact number of instants when performing transient simulation, parsing a simple netlist to define the circuit and the simulation, and generating a CSV format of all node voltages and currents in order to plot the transient graph. It was also vital that the calculations involving node voltages and currents have high precision and consistency.

Some other requirements were considered as non-functional and they ensured the usability and effectiveness of the entire software system. One key requirement was to define classes and data structures that can be extended in the future work. Flexibility is crucial for digital simulation of electronic circuits as additional components or features might be added at a later time. These potential modifications are usually very costly and are likely to corrupt the original version of the software. Therefore, our design has well prepared for any potential modification in the future.

Circuit analysis usually requires a lot of steps and there are many approaches to the same problem. Therefore, choosing an effective algorithm to solve circuit problems is critical. In order to solve circuit problems, Kirchhoff’s current law is frequently used to find the unknown node voltages in a circuit. Hence, new data structures were created to support finding solutions to certain complex networks. A well-known approach is to construct a conductance matrix which is written in the form $G\underline{v} = \underline{i}$, where $G$ is the conductance matrix, $\underline{v}$ and $\underline{i}$ represent columns of node voltages and currents respectively.

In addition, time complexity and space complexity were frequently considered throughout the design process. The execution speed was maximised, and this goal has achieved by choosing efficient algorithms. Also, the memory usage is minimised as pointers are used in this design. Dynamic memory allocation for objects means that the memory assigned to these objects will be released when destructors are called. Indeed, there would be a trade-off between the speed of the algorithm and the memory usage since both cannot be achieved at the same time. In the design, a balanced approach was used and was justified in a critical way. This report will mainly focus on six parts of the design, beginning with project planning on the next section.

