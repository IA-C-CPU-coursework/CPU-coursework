\section{Test bench}

\subsection{Test bench Overview}

This test bench is designed to be able to carry out test on any cpu that satisfies MIPS1 specification and avalon memory-mapped interface. 

The main shell script contains three parts: argument validation, preprocessing on test cases, compilation/execution/comparation of test case on the target cpu.

During argument validation, the main shell script assumes one compulsory argument for source directory of cpu and one optional argument for a specific instruction to test. The shell script would abort either when an invalid source directory or an unsupported instruction is supplied.

Preprocessing of the test cases removes previously generated files if they exist, and generates test cases from `readme.md` file in each instrucion folder. Each test case contains four files: assembly code, v0 reference, initial content for data section, reference for data section.

\begin{figure}
    \centering
    \begin{subfigure}[b]{0.65\textwidth}
        \resizebox{\textwidth}{!}{
            \begin{tikzpicture}[node distance=2cm]
                % define flowchart elements
                \tikzstyle{startstop} = [
                rectangle, 
                align=center, 
                rounded corners, 
                minimum width=3cm, 
                minimum height=1cm,
                text centered, 
                draw=black, 
                fill=red!30
                ]
                \tikzstyle{process} = [
                rectangle, 
                align=center, 
                minimum width=3cm, 
                minimum height=1cm, 
                text centered, 
                draw=black, 
                fill=orange!30,
                execute at begin node=\setlength{\baselineskip}{1em}
                ]
                \tikzstyle{decision} = [diamond, 
                align=center, 
                aspect=2, 
                minimum width=3cm, 
                minimum height=1cm, 
                text centered, 
                draw=black, 
                fill=green!30,
                execute at begin node=\setlength{\baselineskip}{1em}
                ]
                \tikzstyle{io} = [trapezium, 
                trapezium left angle=70, 
                trapezium right angle=110, 
                minimum width=1cm, 
                minimum height=1cm, 
                text centered, 
                draw=black, 
                fill=blue!30,
                execute at begin node=\setlength{\baselineskip}{1em}
                ]
                \tikzstyle{arrow} = [thick,->,>=stealth]

                \node (start) [startstop] {Start};
                \node (arg_handler) [decision, below of=start] {Are\\arguments\\valid?};
                \node (error) [io, left of=arg_handler, xshift=-4.0cm] {print error};
                \node (clean_all) [process, below of=arg_handler, yshift=-0.5cm] {clean previously generated files};
                \node (assembler) [process, below of=clean_all, yshift=0.5cm] {assemble all test cases in\\test/testcases directory};
                \node (instruction_loop) [decision, below of=assembler] {for instrucion\\in instructions};
                \node (testcase_loop) [decision, below of=instruction_loop, yshift=-1.0cm] {for test case\\in test cases};
                \node (compilation) [process, below of=testcase_loop, yshift=-0.5cm] {compile test case\\with iverilog};
                \node (execution) [process, below of=compilation, yshift=0.5cm] {execute test case};
                \node (comparation) [decision, below of=execution, yshift=0cm] {Is result\\the same as\\reference?};
                \node (pass) [io, left of=comparation, xshift=-4.0cm] {print "pass"};
                \node (fail) [io, right of=comparation, xshift=4.0cm] {print "fail"};
                \node (stop) [startstop, left of=instruction_loop, xshift=-3.0cm] {Stop};

                \draw [arrow] (start) -- (arg_handler);
                \draw [arrow] (arg_handler) -- node[anchor=east] {TRUE} (clean_all);
                \draw [arrow] (arg_handler) -- node[anchor=south] {FALSE} (error);
                \draw [arrow] (clean_all) -- (assembler);
                \draw [arrow] (assembler) -- (instruction_loop);
                \draw [arrow] (instruction_loop) -- node[anchor=east] {TRUE} (testcase_loop);
                \draw [arrow] (instruction_loop.west) -- node[anchor=south] {FALSE} (stop.east);
                \draw [arrow] (testcase_loop) -- node[anchor=east] {TRUE} (compilation);
                \draw [arrow] (testcase_loop.east) -- ++(2,0) -- ++(0,1.0) -- ++(0,2) -- node[anchor=east, yshift=-2.5cm, xshift=0.8cm] {FALSE} (instruction_loop.east);
                \draw [arrow] (compilation) -- (execution);
                \draw [arrow] (comparation) -- node[anchor=south] {TRUE} (pass);
                \draw [arrow] (comparation) -- node[anchor=south] {FALSE} (fail);
                \draw [arrow] (execution) -- (comparation);
                \draw [arrow] (comparation.south) -- ++(0,-0.8) -- ++(-9,0) -- ++(0,8) -- (testcase_loop.west);
            \end{tikzpicture}
        }
            \caption{flow chart for test bench}
    \end{subfigure}
    \begin{subfigure}[b]{0.3\textwidth}
        \scriptsize
            \dirtree{%
                .1 test/.
                .2 RAM\_32x64k\_avalon.v.
                .2 generate\_testcases.sh.
                .2 mips\_cpu\_bus\_tb.v.
                .2 test\_mips\_cpu\_bus.sh.
                .2 testcases/.
                .3 instruction\_1.
                .4 testcase\_1.
                .5 testcase.S.
                .5 testcase\_data.init.
                .5 testcase\_data.ref.
                .5 testcase\_v0.ref.
                .4 testcase\_2.
                .4 ....
                .3 instruction\_2.
                .3 ....
            }
            \caption{structure for test/ directory}
            \vfill
    \end{subfigure}
\end{figure}

%\begin{lstlisting}[caption=Parser part 1, label=lst:parser]
%\end{lstlisting}
